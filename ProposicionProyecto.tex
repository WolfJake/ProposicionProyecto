% !TEX TS-program = pdflatex
% !TEX encoding = UTF-8 Unicode

% This is a simple template for a LaTeX document using the "article" class.
% See "book", "report", "letter" for other types of document.

\documentclass[11pt]{article} % use larger type; default would be 10pt

\usepackage[utf8]{inputenc} % set input encoding (not needed with XeLaTeX)

%%% Examples of Article customizations
% These packages are optional, depending whether you want the features they provide.
% See the LaTeX Companion or other references for full information.

%%% PAGE DIMENSIONS
\usepackage{geometry} % to change the page dimensions
\geometry{a4paper} % or letterpaper (US) or a5paper or....
% \geometry{margin=2in} % for example, change the margins to 2 inches all round
% \geometry{landscape} % set up the page for landscape
%   read geometry.pdf for detailed page layout information

\usepackage{graphicx} % support the \includegraphics command and options

% \usepackage[parfill]{parskip} % Activate to begin paragraphs with an empty line rather than an indent

%%% PACKAGES
\usepackage{booktabs} % for much better looking tables
\usepackage{array} % for better arrays (eg matrices) in maths
\usepackage{paralist} % very flexible & customisable lists (eg. enumerate/itemize, etc.)
\usepackage{verbatim} % adds environment for commenting out blocks of text & for better verbatim
\usepackage{subfig} % make it possible to include more than one captioned figure/table in a single float
% These packages are all incorporated in the memoir class to one degree or another...

%%% HEADERS & FOOTERS
\usepackage{fancyhdr} % This should be set AFTER setting up the page geometry
\pagestyle{fancy} % options: empty , plain , fancy
\renewcommand{\headrulewidth}{0pt} % customise the layout...
\lhead{}\chead{}\rhead{}
\lfoot{}\cfoot{\thepage}\rfoot{}

%%% SECTION TITLE APPEARANCE
\usepackage{sectsty}
\allsectionsfont{\sffamily\mdseries\upshape} % (See the fntguide.pdf for font help)
% (This matches ConTeXt defaults)

%%% ToC (table of contents) APPEARANCE
\usepackage[nottoc,notlof,notlot]{tocbibind} % Put the bibliography in the ToC
\usepackage[titles,subfigure]{tocloft} % Alter the style of the Table of Contents
\renewcommand{\cftsecfont}{\rmfamily\mdseries\upshape}
\renewcommand{\cftsecpagefont}{\rmfamily\mdseries\upshape} % No bold!

%%% END Article customizations

%%% The "real" document content comes below...

\title{Propuesta de Proyecto}
\author{Alvaro Ortiz}
%\date{} % Activate to display a given date or no date (if empty),
         % otherwise the current date is printed 

\begin{document}
\maketitle

\section{Introduccion y Justificacion}
\section{Marco Teorico}
\section{Analisis}
\section {Recursos}

PowerWall\\
La Powerwall es una pantalla de pared capaz de proyectar CGI de gran tamaño de forma nitida debido a su muy alta resolucion.

La Powerwall esta compuesta de varias pantallas mas pequeñas de alta resolucion, comunmente de 1600x1200 pixeles, por lo cual conserva su densidad de pixeles sin deteriorar la imagen a pesar de su gran tamaño. Las dimensiones del conjunto de pantallas al unirse suele tener dimensiones entre 3-16 m de largo y 3-4 m de alto.

Manufacturadores de las pantallas ofrecen a sus clientes ciertos niveles de personalizacion, en favor de que las pantallas se acoplen con mayor facilidad a las aplicaciones que se les quiera dar.\\

3D Glasses\\
Las gafas de 3D que se utilizan son de estereoscopia tridimensional activa los cuales, se diferencian del resto por tener componentes electronicos que interactuan con la imagen. Esta funciona presentando completamente la imagen destinada al ojo derecho mientras que bloquea la del ojo izquierdo y viceversa. Esta presentacion se repite con una freecuencia tan alta que el usuario percibe la fusion de las dos imagenes 2D como una sola iamgen 3D sin ser interrumpido por el cambio.\\

IR Tracker\\


\end{document}
